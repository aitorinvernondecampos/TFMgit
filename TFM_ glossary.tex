%%%%%%%%%%%%%%%%
% Glosario TFM %
%%%%%%%%%%%%%%%%

\makeglossaries

% Definición de términos
%%%%%%%%%%%%%%%%%%%%%%%%

% \newglossaryentry{ex}{name={sample},description={an example}}

\newglossaryentry{estadística multivariante}
  {name=estadística multivariante,
   description={La estadística multivariante o multivariada es una rama de las estadísticas que abarca la observación y el análisis simultáneos de más de una variable respuesta. La aplicación de la estadística multivariante es llamada análisis estadístico multivariante}
  }
  
\newglossaryentry{MANOVA}
  {name=MANOVA,
   description={En estadística el análisis multivariante de la varianza (Multivariate analysis of variance) es una extensión del análisis de la varianza o ANOVA para cubrir los casos donde hay más de una variable dependiente que no pueden ser combinadas de manera simple. Además de identificar si los cambios en las variables independientes tienen efectos significativos en las variables dependientes, la técnica también intenta identificar las interacciones entre las variables independientes y su grado de asociación con las dependientes}
  }

\newglossaryentry{PERMANOVA}
  {name=PERMANOVA,
   description={El análisis multivariante de la varianza con permutaciones (Permutational multivariate analysis of variance, PERMANOVA) es una prueba de permutación estadística multivariada no paramétrica. Se utiliza para comparar grupos de objetos y probar la hipótesis nula de que los centroides y la dispersión de los grupos definidos por el espacio de medida son equivalentes para todos los grupos. Un rechazo de la hipótesis nula significa que el centro y/o la dispersión de los objetos es diferente entre los grupos. De esta manera, la prueba se basa en el cálculo previo de la distancia entre cualesquier dos objetos incluidos en el experimento}
  }
  
\newglossaryentry{SNPs}
  {name=SNPs,
   description={Un polimorfismo puntual, también denominado de un solo nucleótido o SNP (Single Nucleotide Polymorphism, pronunciado snip), es una variación en la secuencia de ADN que afecta a una sola base (adenina (A), timina (T), citosina (C) o guanina (G)) de una secuencia del genoma. Sin embargo, generalmente se considera que cambios de unos pocos nucleótidos, como también pequeñas inserciones y deleciones (indels) pueden ser consideradas como SNP. Una de estas variaciones debe darse al menos en un 1 \% de la población para ser considerada como un SNP. Si no se llega al 1 \% no se considera SNP y sí una mutación puntual. En ocasiones estas variaciones de nucleótido único se asocian a otro término conocido como SNV (Single Nucleotide Variant), que a diferencia de los SNPs carece de limitaciones de frecuencia}
  }
  
\newglossaryentry{GWAS}
  {name=GWAS,
   description={En genética, un estudio de asociación del genoma completo (Genome-wide association study) o WGAS (Whole genome association study) es un análisis de una variación genética a lo largo de todo el genoma humano con el objetivo de identificar su asociación a un rasgo observable. Los GWAS suelen centrarse en asociaciones entre los polimorfismos de un solo nucleótido (SNPs) y rasgos como las principales enfermedades}
  }
  
\newglossaryentry{sQTLs}
  {name=sQTLs,
   description={Los \textit{Splicing quantitative trait loci} (sQTLs o splicing QTLs) son los loci que regulan el splicing alternativo del ARNm. Se pueden detectar utilizando datos de RNA-seq. Se han desarrollado diversos métodos para descubrir sQTLs, entre los que se incluyen: LeafCutter, Altrans, Cufflinks, y MISO}
  }
  
\newglossaryentry{RNA-seq}
  {name=RNA-seq,
   description={La secuenciación de ARN, también llamada \textit{Secuenciación del Transcriptoma Entero para Clonación al Azar}, utiliza la secuenciación masiva (NGS) para revelar la presencia y cantidad de ARN, en una muestra biológica en un momento dado. De esta manera, la RNA-seq se usa para analizar cambios en el transcriptoma, concretamente, facilita la observación de transcritos resultantes del empalme alternativo, modificación postranscripcional, fusiones génicas, mutaciones/polimorfismos de nucleótidos únicos y cambios de expresión de genes. Puede ayudar a caracterizar poblaciones diferentes de RNA como miRNA, tRNA, y rRNA, o para determinar las fronteras exón/intrón y verificar o enmendar regiones 5' y 3'}
  }

\newglossaryentry{trait}
  {name=trait,
   description={En el ámbito de la genética, un \textit{trait} o \textit{rasgo} es una característica específica de un individuo, los cuales pueden ser determinados por genes, factores ambientales o por una combinación de ambos. Se clasifican como cualitativos (e.g. el color de los ojos) o cuantitativos (e.g. la altura o la presión sanguínea). Cada uno de ellos forma parte del fenotipo general de un individuo}
  }

\newglossaryentry{pleiotropía}
  {name=pleiotropía,
   description={En biología, la pleiotropía o polifenia es el fenómeno por el cual un solo gen o alelo es responsable de efectos fenotípicos o caracteres distintos y no relacionados (e.g. la fenilcetonuria, la talasemia o anemia de células falciformes, o el albinismo de los animales que tiene un efecto pleitrópico en sus emociones haciéndolos más reactivos a su entorno)}
  }

\newglossaryentry{MANTA}
  {name=MANTA,
   description={Multivariate Asymptotic Non-parametric Test of Association. Este paquete, programado en lenguaje R, permite el cálculo no paramétrico y asimptótico del p-valor para modelos lineales multivariados}
  }
  
\newglossaryentry{MultiPhen}
  {name=MultiPhen,
   description={Paquete de R que permite testear la asociación de múltiples rasgos. Realiza pruebas de asociación genética entre SNPs y múltiples fenotipos (por separado o en conjunto)}
  }

\newglossaryentry{mvLMMs}
  {name=mvLMMs,
   description={Los modelos lineales mixtos multivariados son poderosas herramientas para probar asociaciones entre polimorfismos de núcleo único y múltiples fenotipos correlacionados mientras controlan la estratificación de la población en estudios de asociación de todo el genoma}
  }

\newglossaryentry{MTAR}
  {name=MTAR,
   description={Marco desarrollado para el análisis multi-trait de RVAS. Se basa en un meta-modelo analítico de efectos aleatorios que utiliza diferentes estructuras de correlación de los efectos genéticos para representar un amplio espectro de patrones de asociación a través de rasgos y variantes}
  }

\newglossaryentry{MOSTest}
  {name=MOSTest,
   description={Es una herramienta para unir el análisis genético de múltiples rasgos, que utiliza el análisis multivariado para aumentar la potencia, y así poder descubrir los loci asociados}
  }

\newglossaryentry{error de tipo I}
  {name=error de tipo I,
   description={En un estudio de investigación, el error de tipo I, también denominado error de tipo alfa (\alpha) o falso positivo, es el error que se comete cuando el investigador rechaza la hipótesis nula (*H*$_{0}$: el supuesto inicial) siendo esta verdadera en la población. Es equivalente a encontrar un resultado falso positivo, porque el investigador llega a la conclusión de que existe una diferencia entre las hipótesis cuando en realidad no existe. Se relaciona con el nivel de significancia estadística}
  }

\newglossaryentry{error de tipo II}
  {name=error de tipo II,
   description={En un estudio de investigación, el error de tipo II, también llamado error de tipo beta (donde \beta es la probabilidad de que exista este error) o falso negativo, se comete cuando el investigador no rechaza la hipótesis nula (*H*$_{0}$: el supuesto inicial) siendo esta falsa en la población. Es equivalente a la probabilidad de un resultado falso negativo, ya que el investigador llega a la conclusión de que ha sido incapaz de encontrar una diferencia que existe en la realidad. De forma general y dependiendo de cada caso, se suele aceptar en un estudio que el valor del error beta esté entre el 5 y el 20 \%}
  }

\newglossaryentry{potencia}
  {name=potencia,
   description={La potencia o poder de una prueba estadística es la probabilidad de que la hipótesis alternativa sea aceptada cuando la hipótesis alternativa es verdadera, es decir, la probabilidad de no cometer un error del tipo II. En general, es una función de las distribuciones posibles, a menudo determinada por un parámetro, bajo la hipótesis alternativa. A medida que aumenta la potencia, las posibilidades de que se presente un error del tipo II se reducen (disminución de la tasa de falsos negativos \beta), de esta manera, la potencia se representa como $1-\beta$ (sensibilidad). El análisis de la potencia se puede utilizar para calcular el tamaño mínimo de la muestra necesario para que uno pueda detectar razonablemente un efecto de un determinado tamaño, o también para calcular el tamaño del efecto mínimo que es probable que se detecte en un estudio usando un tamaño de muestra dado. Además, el concepto de \textit{alimentación} se utiliza para hacer comparaciones entre diferentes procedimientos de análisis estadísticos (e.g. entre una prueba paramétrica y una no paramétrica de la misma hipótesis}
  }

\newglossaryentry{pseudo-F}
  {name=pseudo-F,
   description={En el análisis multivariante de la varianza con permutacionesa (\textit{PERMANOVA}), el estadístico de prueba es una pseudo-ratio F, similar a la relación F en ANOVA. Compara la suma total de diferencias cuadradas (o diferencias de orden) entre objetos pertenecientes a diferentes grupos con la de objetos que pertenecen al mismo grupo. Las F-ratios más grandes indican una separación de grupo más pronunciada, sin embargo, la significación estadística de esta relación suele ser más interesante que su magnitud}
  }

\newglossaryentry{valores p}
  {name=valores p,
   description={En estadística general y contrastes de hipótesis, los valores p (p, p-valor, valor de p consignado, o p-value) se define como la probabilidad de que un valor estadístico calculado sea posible dada una hipótesis nula cierta. Ayuda a diferenciar resultados que son producto del azar del muestreo, de resultados que son estadísticamente significativos. Alternativamente, se define como la probabilidad de observar los resultados del estudio, u otros más alejados de la hipótesis nula, si la hipótesis nula fuera cierta, de manera que si este cumple con la condición de ser menor que un nivel de significancia impuesto arbitrariamente, este se considera como un resultado estadísticamente significativo y, por lo tanto, permite rechazar la hipótesis nula}
  }


% ==> SEGUIR CON GLOSARIO A PARTIR DEL ESTADO DEL ARTE
  
  
% ------------------------------------------------------------------------------------------------------------------------------------------------

% Cómo especificar los términos del glosario en el documento LATEX:

% \gls{ }
% To print the term, lowercase. For example, \gls{maths} prints mathematics when used.

% \Gls{ }
% The same as \gls but the first letter will be printed in uppercase. Example: \Gls{maths} prints Mathematics

% \glspl{ }
% The same as \gls but the term is put in its plural form. For instance, \glspl{formula} will write formulas in your final document.

% \Glspl{ }
% The same as \Gls but the term is put in its plural form. For example, \Glspl{formula} renders as Formulas.


% Estilos del glosario:

% The command \glossarystyle{style} must be inserted before \printglossaries. Below a list of available styles:

% list. Writes the defined term in boldface font
% altlist. Inserts newline after the term and indents the description.
% listgroup. Group the terms based on the first letter.
% listhypergroup. Adds hyperlinks at the top of the index.


% Imprimir el glosario en el documento:
% 
% \printglossary


% Cómo especificar los acrónimos en el documento LATEX:

% \acrlong{ }
% Displays the phrase which the acronyms stands for. Put the label of the acronym inside the braces. In the example, \acrlong{gcd} prints Greatest Common Divisor.

% \acrshort{ }
% Prints the acronym whose label is passed as parameter. For instance, \acrshort{gcd} renders as GCD.

% \acrfull{ }
% Prints both, the acronym and its definition. In the example the output of \acrfull{lcm} is Least Common Multiple (LCM).

% To print the list of acronyms use the command
% 
% \printglossary[type=\acronymtype]


% Cambiar el nombre del glosario y del título en TOC:

% \printglossary[title=Special Terms, toctitle=List of terms]

% Notice that the command \printglossary has two comma-separated parameters:
% title=Special Terms is the title to be displayed on top of the glossary.
% toctitle=List of terms this is the entry to be displayed in the table of contents. 
